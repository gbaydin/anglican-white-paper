\documentclass[a4paper]{article}

\usepackage{enumitem}
\usepackage{amsmath}
\usepackage{amssymb}
\usepackage{amsfonts}
\usepackage{graphicx}

\usepackage{algorithm}
\usepackage{algorithmic}
\usepackage{listings}

\usepackage{hyperref}
\usepackage{url}

\title{Notes on Design and Implementation of Anglican}

\begin{document}

\maketitle

\begin{abstract}
	In these notes, are explain the rationale behind the
	macro-compiled implementation of Anglican, outline
	design choices, and describe important implementation
	aspects.
\end{abstract}

\section{Introduction}

Anglican is a probabilistic programming language integrated with
Clojure.  There are several ways to build a programming language
on top or besides another language.  The easiest to grasp is an
interpreter --- a program that reads a program, in its entirety
or line-by-line, and executes it by applying operational
semantics of a certain kind to the language. \textsc{Basic} is
famous for line-by-line interpreted implementations.

Another approach is to write a compiler, either to a virtual
architecture, so called p-code or byte-code, or to real
hardware. Here, the whole program is translated from the
`higher-level' source language to a `lower-level' object
language, which can be directly executed, either by hardware or
by an interpreter --- but the latter interpreter can be made
simpler and more efficient  than an interpreter for the source
language.

On top of these two approaches are methods in which a new
language is implemented `inside' another language of the same
level of abstraction. Different languages provide different
means for this; Lisp is notorious for the macro facility
that allows to extend the language almost without
restriction --- by writing \textit{macros}, one adds new
constructs to the existing language. There are several uses of
macros --- one is to extend the language \textit{syntax}, for
example, by adding new control structures; another is to keep
the existing syntax but alter the \textit{operational semantics}
--- the way programs are executed and compute their outputs.

Anglican is implemented in just this way --- a macro facility
provided by Clojure, a Lisp dialect, is used both to extend
Clojure with constructs that delimit probabilistic code, and to
alter the operational semantics of Clojure expressions inside
probabilistic code fragments. Anglican claims its right to count
as a separate language because of the ubiquitous probabilistic
execution semantics rather than a different syntax,
which is actually an advantage rather than a drawback ---
Clojure programmers only need to know how to specify the
boundaries of Anglican programs, but can use familiar Clojure
syntax to write probabilistic code. 

An implementation of Anglican must therefore address three issues:
\begin{itemize}
	\item the Clojure syntax to introduce probabilistic Anglican
		code inside Clojure modules;
	\item source-to-source transformation of Anglican programs
		into Clojure, so that probabilistic execution becomes
		possible;
	\item algorithms which run Clojure code, obtained by
		transforming Anglican programs, according to the
		probabilistic operational semantics.
\end{itemize}
The following sections explain the way Anglican is implemented,
from source-to-source transformation and syntactic wrappers to
inference algorithms which accept Clojure functions built from
Anglican code as a parameter, and produce probabilistic outputs.

\section{Design Outline}

A probabilistic program, or query, mostly runs deterministic
code, except for certain checkpoints, in which probabilities are
involved, and normal, linear execution of the program is
disrupted. In Anglican and similar languages there are two
types of such checkpoints:
\begin{itemize}
	\item drawing a value from a random source (\texttt{sample});
	\item conditioning the posterior distribution by
		conditioning a computed value on a random source
		(\texttt{observe}).
\end{itemize}

Anglican can be mostly implemented as a regular programming
language, except for the handling of these checkpoints.
Depending on the \textit{inference algorithm}, \texttt{sample}
and \texttt{observe} may result in implicit input/output
operations and control changes. For example, \texttt{observe} in
particle filtering inference algorithms is a non-deterministic
control statement at which a particle can be either replicated
or terminated. Similarly, in Metropolis-Hastings,
\texttt{sample} is both an input and a non-deterministic control
statement (with delayed effect), eventually affecting acceptance
or rejection of a sample.

Because of the checkpoints, Anglican programs must allow the
inference algorithm to step in, recording information and
affecting control flow. This can be implemented through
coroutines/cooperative multitasking, parallel execution/preemptive
multitasking and shared memory, as well as through explicit
maintenance of program continuations at checkpoints. Clojure is
a functional language, and continuation-passing style (CPS)
transformation is a well-developed technique in the area of
functional languages. Implementing a variant of CPS
transformation seemed to be the most flexible and lightweight
option --- any other form of concurrency would put a higher burden
on the underlying runtime (JVM) and the operating system.
Consequently, Anglican has been implemented as a CPS-transformed
computation with access to continuations in probabilistic
checkpoints. Anglican `compiler', represented by a set of
functions in the \texttt{anglican.trap} namespace, accepts a
Clojure subset and transforms it into a variant of CPS
representation, which allows inference algorithms to intervene
in the execution flow at probabilistic checkpoints.

Anglican is intended to co-exist with Clojure and be a part of
the source of a Clojure program. To facilitate this, Anglican
programs, or queries, are wrapped by macros (defined in the
\texttt{anglican.emit} namespace), which call the CPS
transformations and define Clojure objects suitable for passing
as arguments to inference algorithms (\texttt{defquery},
\texttt{query}). In addition to defining entire queries,
Anglican promotes modularization of inference algorithms through
definition of \textit{probabilistic functions} using
\texttt{defm} and \texttt{fm} (Anglican counterparts of Clojure
\texttt{defn} and \texttt{fn}). Probabilistic functions are
written in Anglican, may include probabilistic forms
\texttt{sample} and \texttt{observe} (as well \texttt{predict}
for the output), and can be seamlessly called from inside
Anglican queries, just like functions locally defined within the
same query.

Operational semantics of Anglican queries is different from that
of Clojure code, therefore queries must be called through
inference algorithms, rather than `directly'.  The
\texttt{anglican.inference} namespace supplies the
\texttt{infer} multimethod, which accepts an Anglican query and
returns a lazy sequence of weighted samples from the
distribution defined by the query.  When inference
is performed on an Anglican query, the query is run by a
particular inference algorithm. Inference algorithms must
provide an implementation for \texttt{infer}, as well as
override some of the methods of the \texttt{checkpoint}
multimethod, called to handle \texttt{sample} and
\texttt{observe} in an algorithmic-specific manner, as well
as on termination of a probabilistic program.

Finally, Anglican queries use `primitive', or commonly known
and used, distributions, to draw random samples and condition
observations. Many primitive distributions are provided by the
\texttt{anglican.runtime} namespace, and additional
distributions can be defined by the user by implementing the
\texttt{distribution} protocol. The \texttt{defdist} macro
provides a convenient syntax for defining primitive distributions.

Compilation, invocation, and runtime support of Anglican queries
are discussed in detail in further sections.

\section{Macro-based Compilation}

Compilation of Anglican into Clojure relies on the Clojure
\textit{macro} facility. However, the compilation algorithm is
implemented as a library of \textit{functions} in namespace
\texttt{anglican.trap}, which are invoked by macros. Namespace
\texttt{anglican.trap-test} contains many transformation tests,
which both help assure proper functioning of the transformation
functions, and serve as illustrative examples of individual
transformations, such as of functions, flow control, and
probabilistic contstructs.

The CPS transformation is organized in top-down manner. The
top-level function is  \texttt{cps-of-expression}, which
receives an expression and a continuation, and returns the
expression in the CPS form, with the computed result passed to
the continuation. A continuation accepts \texttt{two arguments}:
\begin{itemize}
	\item the computed value;
	\item the internal state, bound to the local variable
		\texttt{\$state} in every lexical scope.
\end{itemize}
The state (\texttt{\$state}) is threaded through the computation
and contains data used by inference algorithm. \texttt{\$state}
is a Clojure hash map, and the map entries are
algorithm-dependent. Except for transformation of
\texttt{mem}, the memoization form, CPS transformation routines
are not aware of contents of \texttt{\$state}, do not access or
modify it directly, but rather just thread the state unmodified
through the computation. Algorithm-specific handlers of
checkpoints corresponding to the probabilistic forms
(\texttt{sample}, \texttt{observe}) modify the
state and reinject a new state into the computation.

\subsection{Expression Kinds}

There are three different kinds of inputs to CPS transformation:
\begin{itemize}
	\item literals, which are passed as an argument to the
		continuation unmodified;
	\item value expressions (e.g. the \texttt{fn} form) 
		(called \textit{opaque}
		expressions in the code) which must be transformed to
		CPS, but the transformed object is passed to the
		continuation as a whole, opaquely;
	\item general expressions (let's call them
		\textit{transparent}), through which the continuation is
		threaded in an expression-specific way, and can be
		called in multiple locations of the CPS-transformed
		code, such as in all branches of an \texttt{if}
		statement.
\end{itemize}

\subsubsection{Literals}

Literals are the same in Anglican and Clojure. They are 
left unmodified. Literals are a subset of opaque expressions,
and are identified by test \texttt{simple?} called from
\texttt{opaque?}.  However, The Clojure syntax has a peculiarity
of using the syntax of compound literals (vectors, hash maps,
and sets) for data constructors. Hence, compound literals must
be traversed recursively, and if there is a nested non-literal
component, transformed into a call to the corresponding data
constructor. Functions \texttt{cps-of-vector},
\texttt{cps-of-hash-map}, \texttt{cps-of-set}, called from
\texttt{cps-of-expression}, transform Clojure constructor syntax
(\texttt{[...]}, \texttt{\{...\}}, \texttt{\#\{...\}}) into the
corresponding calls.

\subsubsection{Opaque Expressions}

Opaque, or value, expressions, have a different shape in the
original and the CPS form. However, their CPS form follows the
pattern \texttt{(continuation transformed-expression)}, and thus
the transformation does not depend on the continuation, and
can be accomplished without passing the continuation as a
transformation argument. Primitive (non-CPS) procedures used in
Anglican code, \texttt{(fn ...)} forms, and \text{(mem ...)}
forms are opaque and transformed by
\texttt{primitive-procedure-cps}, \texttt{fn-cps}, and
\texttt{mem-cps}, correspondingly.

\subsubsection{General Expressions}

The most general form of CPS transformation receives an
expression and a continuation, and returns the expression
in CPS form with the continuation potentially called in multiple
tail positions. General expressions can be somewhat voluntarily
divided into several groups:
\begin{itemize}
	\item binding forms --- \texttt{let} and
		\texttt{loop/recur};
	\item flow control --- \texttt{if}, \texttt{when},
		\texttt{cond}, \texttt{case}, \texttt{and}, \texttt{or} and \texttt{do};
	\item function applications and \texttt{apply};
	\item probabilistic forms --- \texttt{predict},
		\texttt{observe}, \texttt{sample}, \texttt{store}, and \texttt{retrieve}.
\end{itemize}
Functions that transform general expressions accept the
expression and the continuation as parameters, and are
consistently named \texttt{cps-of-}\textit{form}, for example,
\texttt{cps-of-do}, \texttt{cps-of-store}.

\subsection{Implementation Highlights}

\subsubsection{Continuations}

\subsubsection{Primitive Procedures}

\subsubsection{Probabilistic forms}

\subsubsection{Stochastic memoization}

\section{Inference Algorithms}

\section{Definitions and Runtime Library}


\bibliographystyle{alpha}
\bibliography{refs}

\end{document}
