\documentclass[a4paper]{article}

\usepackage{enumitem}
\usepackage{amsmath}
\usepackage{amssymb}
\usepackage{amsfonts}
\usepackage{graphicx}

\usepackage{algorithm}
\usepackage{algorithmic}
\usepackage{listings}

\usepackage{hyperref}
\usepackage{url}

\title{Notes on Design and Implementation of Anglican}

\begin{document}

\maketitle

\begin{abstract}
	In these notes, are explain the rationale behind the
	macro-compiled implementation of Anglican, outline
	design choices, and describe important implementation
	aspects.
\end{abstract}

\section{Introduction}

Anglican is a probabilistic programming language integrated with
Clojure.  There are several ways to build a programming language
on top or besides another language.  The easiest to grasp is an
interpreter --- a program that reads a program, in its entirety
or line-by-line, and executes it by applying operational
semantics of a certain kind to the language. \textsc{Basic} is
famous for line-by-line interpreted implementations.

Another approach is to write a compiler, either to a virtual
architecture, so called p-code or byte-code, or to real
hardware. Here, the whole program is translated from the
`higher-level' source language to a `lower-level' object
language, which can be directly executed, either by hardware or
by an interpreter --- but the latter interpreter can be made
simpler and more efficient  than an interpreter for the source
language.

On top of these two approaches are methods in which a new
language is implemented `inside' another language of the same
level of abstraction. Different languages provide different
means for this; Lisp is notorious for the macro facility
that allows to extend the language almost without
restriction --- by writing \textit{macros}, one adds new
constructs to the existing language. There are several uses of
macros --- one is to extend the language \textit{syntax}, for
example, by adding new control structures; another is to keep
the existing syntax but alter the operational semantic --- the
way programs are executed and compute their outputs.

Anglican is implemented in just this way --- a macro facility
provided by Clojure, a Lisp dialect, is used both to extend
Clojure with constructs that delimit probabilistic code, and to
alter the operational semantics of Clojure expressions inside
probabilistic code fragments. Anglican claims its right to count
as a separate language because of the ubiquitous probabilistic
execution semantics rather than because of a different syntax,
which is actually an advantage rather than a drawback ---
Clojure programmers only need to know how to specify the
boundaries of Anglican programs, but can use familiar Clojure
syntax to write probabilistic code. 

An implementation of Anglican must therefore address three issues:
\begin{itemize}
	\item the Clojure syntax to introduce probabilistic Anglican
		code inside Clojure modules;
	\item source-to-source transformation of Anglican programs
		into Clojure, so that probabilistic execution becomes
		possible;
	\item algorithms which run Clojure code, obtained by
		transforming Anglican programs, according to the
		probabilistic operational semantics.
\end{itemize}
The following sections explain the way Anglican is implemented,
from source-to-source transformation and syntactic wrappers to
inference algorithms which accept Clojure functions built from
Anglican code as a parameter, and produce probabilistic outputs.

\section{Design Outline}

A probabilistic program, or query, mostly runs deterministic
code, except for certain checkpoints, in which probabilities are
involved, and normal, linear execution of the program is
disrupted. In Anglican and similar languages there are two
types of such checkpoints:
\begin{itemize}
	\item drawing a value from a random source (\texttt{sample});
	\item conditioning the posterior distribution by
		conditioning a computed value on a random source
		(\texttt{observe}).
\end{itemize}

Anglican can be mostly implemented as a regular programming
language, except for the handling of these checkpoints.
Depending on the \textit{inference algorithm}, \texttt{sample}
and \texttt{observe} may result in implicit input/output
operations and control changes. For example, \texttt{observe} in
particle filtering inference algorithms is a non-deterministic
control statement at which a particle can be either replicated
or terminated. Similarly, in Metropolis-Hastings,
\texttt{sample} is both an input and a non-deterministic control
statement (with delayed effect), eventually affecting acceptance
or rejection of a sample.

Because of the checkpoints, Anglican programs must allow the
inference algorithm to step in, recording information and
affecting control flow. This can be implemented through coroutines/
cooperative multitasking, parallel execution/preemptive
multitasking and shared memory, as well as through explicit
maintenance of program continuations at checkpoints. Clojure is
a functional language, and continuation-passing style (CPS)
transformation is a well-developed technique in the area of
functional languages. Implementing a variant of CPS
transformation seemed to be the most flexible and lightweight
option --- any other form of concurrency would put a higher burden
on the underlying runtime (JVM) and the operating system.
Consequently, Anglican has been implemented as a CPS-transformed
computation with access to continuations in probabilistic
checkpoints. Anglican `compiler', represented by a set of
functions in the \texttt{anglican.trap} namespace, accepts a
Clojure subset and transforms it into a variant of CPS
representation, which allows inference algorithms to intervene
in the execution flow at probabilistic checkpoints.

Anglican is intended to co-exist with Clojure and be a part of
the source of a Clojure program. To facilitate this, Anglican
programs, or queries, are wrapped by macros (defined in the
\texttt{anglican.emit} namespace), which call the CPS
transformations and define Clojure objects suitable for passing
as arguments to inference algorithms (\texttt{defquery},
\texttt{query}). In addition to defining entire queries,
Anglican promotes modularization of inference algorithm through
definition of \textit{probabilistic functions} using
\texttt{defm} and \texttt{fm} (Anglican counterparts of Clojure
\texttt{defn} and \texttt{fn}), which are written in Anglican,
may include probabilistic forms \texttt{sample} and
\texttt{observe} (as well \texttt{predict} for the output), and
can be seamlessly called from inside Anglican queries, just like
functions locally defined within the same query.

Operational semantic of Anglican queries is different from that
of Clojure code, therefore queries must be called through
inference algorithms, rather than `directly'.  The
\texttt{anglican.inference} namespace supplies the
\texttt{infer} multimethod, which accepts an Anglican query and
returns a lazy sequence of weighted samples from the
distribution defined by the query.  When inference
is performed on an Anglican query, the query is run by a
particular inference algorithm. Inference algorithms must
provide an implementation for \texttt{infer}, as well as
override some of the methods of the \textit{checkpoint}
multimethod, called to handle \texttt{sample} and
\texttt{observe} in an algorithmic-specific manner, as well
as on termination of a probabilistic program.

Finally, Anglican queries use `primitive', or commonly known
and used, distributions, to draw random samples and condition
observations. Many primitive distributions are provided by the
\texttt{anglican.runtime} namespace, and additional
distributions can be defined by the user by implementing the
\texttt{distribution} protocol. The \texttt{defdist} macro
provides a convenient syntax for defining primitive distributions.

Compilation, invocation, and runtime support of anglican queries
are discussed in detail in the following sections.

\section{Macro-based Compilation}

\section{Inference Algorithms}

\section{Definitions and Runtime Library}


\bibliographystyle{alpha}
\bibliography{refs}

\end{document}
